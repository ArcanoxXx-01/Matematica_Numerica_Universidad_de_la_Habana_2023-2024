\documentclass[a4paper,12pt]{article}

\begin{document}
\title{CP: 5}
\author{Dario Lopez Falcon}
\date{febrero, 2024}
\maketitle

\section*{Ejercicio \#1( 90000 boniatos)}



\section*{Ejercicio \#2(130000 boniatos)}

Demuestre que $ \|B\|<1\Rightarrow $la sucesion $\{x_0,x_1,...\} $obtenida a partir del sistema iterativo converge a la solucion de sistema 

1) Lemma: Si $p(T)<1\Rightarrow (I-T)^{-1}$ existe y :
\[(I-T)^{-1}=I+T^1+T^2+...=\sum_{k=0}^{\infty}T^k \Rightarrow \Rightarrow (I-T)x=(1-\lambda)x\]
Como tenemos que $\lambda$ es un valor caracteristico de T y si $1-\lambda$ es el valor caracteristico de (I-T) 
    Como $Tx=\lambda x \Rightarrow x-Tx=x-\lambda x$
\section*{Ejercicio \#3(150000 boniatos)}

\section*{Ejercicio \#4(150000 boniatos)}

\section*{Ejercicio \#5(200000 boniatos)}

\section*{Ejercicio \#6(200000 boniatos)}

\section*{Ejercicio \#7(200000 boniatos)}

\section*{Ejercicio \#8(500000 boniatos)}

\section*{Pregunta Secreta(30000 boniatos)}

R/ El Ejercicio 8.

\bibliography{bibliography}
\end{document}